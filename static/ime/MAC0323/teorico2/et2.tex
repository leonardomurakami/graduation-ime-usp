\documentclass[11pt,reqno,a4paper]{amsart}

\usepackage{setspace}
\usepackage[portuguese]{babel}
\usepackage[utf8]{inputenc} % Or use \usepackage{inputenc}[utf8] if needed by editor

\usepackage{fullpage}
\usepackage{amsmath}
\usepackage{amssymb} % For symbols like \lfloor, \rfloor, \min, \max
\usepackage{amsthm} % Recommended for theorem environments
\usepackage{enumitem} % Added for custom list labels

% Consistent spacing for displayed math
\usepackage{mathtools} 
\mathtoolsset{showonlyrefs=false} % Optional: only number referenced equations

\def\Assin#1{\noindent\textit{Assinatura}\strut\\%:\\
\framebox[\textwidth]{\phantom{\vrule height#1}}}

\newtheoremstyle{prova_style}{}{}{\itshape}{}{\bfseries}{.}{ }{}
\theoremstyle{prova_style}
\newtheorem*{prova}{Prova} % Unnumbered proof environment

% Define lg for logarithm base 2
\DeclareMathOperator{\lgop}{lg}
\newcommand{\lgfn}[1]{\lgop #1}

\title{\textsl{
    MAC0323 Algoritmos e Estruturas de Dados II}\\\vspace{3\jot}
  Folha de solução}
\author[MAC0323 Folha de solução]{} % Author is optional or can be filled

\date{13 de Abril de 2025} % Use \date command

\begin{document}
\parindent=0pt % No paragraph indentation, paragraphs separated by vertical space

\maketitle
\thispagestyle{empty} 
\pagestyle{plain}
\onehalfspace % 1.5 line spacing

\textbf{Nome:} Leonardo Heidi Almeida Murakami\enspace \hfill \enspace
\textbf{NUSP:} 11260186\enspace\hfill

\bigskip

\textbf{Exercício:} Exercício Teórico II - MAC0323\enspace\hfill\enspace

\bigskip
\noindent\textbf{SOLUÇÃO}

\section{Exercício 2}

\subsection{Algoritmo Proposto}

\begin{enumerate}[label=(\arabic*)] 
    \item \textbf{Escolha de um vértice raiz arbitrário:} Selecione um vértice arbitrário $s \in V$. Este vértice será o último na ordenação de remoção, ou seja, $x_n = s$.
    \item \textbf{Busca em Profundidade (DFS) e Pós-Ordem:} Realize uma DFS em $G$ iniciando pelo vértice $s$. Durante a travessia, compute uma ordenação de pós-ordem (post-order traversal) dos vértices. Denotaremos esta sequência como $P_1, P_2, \dots, P_n$. Pela definição de pós-ordem em uma DFS iniciada em $s$, temos $P_n = s$.
    \item \textbf{Definição da ordenação de remoção:} A ordenação de remoção desejada $x_1, x_2, \dots, x_n$ é definida diretamente pela sequência de pós-ordem: $x_i = P_i$ para $i=1, \dots, n$.
\end{enumerate}

\subsection{Demonstração}

Precisamos demonstrar que para a ordenação $x_1, \dots, x_n$ gerada, os grafos $G_k = G - \{x_1, \dots, x_k\}$ são conexos para $k=0, 1, \dots, n-1$. O grafo $G_k$ é o subgrafo de $G$ induzido pelo conjunto de vértices $V'_k = V \setminus \{x_1, \dots, x_k\}$. Utilizando a definição $x_i = P_i$, temos $V'_k = \{P_{k+1}, \dots, P_n\}$.

\begin{itemize}
    \item Para $k=0$, $G_0 = G[V \setminus \emptyset] = G[V] = G$. O problema estipula que $G$ é conexo, então a condição é satisfeita para $G_0$.
    \item Para $k \in \{1, \dots, n-1\}$, consideremos o grafo $G_k = G[V'_k]$. Seja $T$ a árvore DFS gerada pela busca em profundidade iniciada em $s=P_n$.
    
    Para qualquer vértice $v \in V'_k$:
    \begin{itemize}
        \item Se $v=s$ (ou seja, $v=P_n$), ele é a raiz da árvore DFS $T$ e pertence a $V'_k$. 
        \item Se $v \neq s$, então $v = P_j$ para algum $j \in \{k+1, \dots, n-1\}$. Na árvore DFS $T$, $v$ possui um pai, que denotaremos $\text{pai}(v)$. De acordo com as propriedades da travessia em pós-ordem, um nó é registrado na sequência de pós-ordem somente após todos os seus descendentes na árvore DFS terem sido registrados. Isso implica que o índice de pós-ordem de qualquer nó é menor que o índice de pós-ordem de seu pai. Assim, se $v = P_j$, seu pai $\text{pai}(v)$ será $P_l$ para algum $l > j$. Como $j \ge k+1$, segue que $l > j \ge k+1$, e portanto $l \in \{k+1, \dots, n\}$. Logo, $\text{pai}(v) \in V'_k$.
    \end{itemize}
    Isso demonstra que cada vértice $v \in V'_k \setminus \{s\}$ está conectado ao seu pai $\text{pai}(v)$ (que também está em $V'_k$) através de uma aresta da árvore $T$. Como $s=P_n$ é a raiz de $T$ e $s \in V'_k$, todos os vértices em $V'_k$ estão conectados a $s$ dentro de $G[V'_k]$ por meio de caminhos formados por arestas de $T$ (que são um subconjunto das arestas de $G[V'_k]$). Consequentemente, o grafo $G[V'_k]$ é conexo.
    
    Esta argumentação é válida para $k=1, \dots, n-1$. Note que para $k=n-1$, $V'_{n-1} = \{P_n\} = \{s\}$. O grafo $G[\{s\}]$ consiste em um único vértice e é, por definição, conexo.
\end{itemize}
Portanto, o algoritmo produz uma ordenação que satisfaz as condições do problema.

\subsection{Análise da Eficiência}

A complexidade do algoritmo é determinada pelas suas etapas principais:
\begin{enumerate}[label=(\arabic*)]
    \item \textbf{Escolha de um vértice raiz arbitrário:} A seleção de um vértice arbitrário pode ser feita em $O(1)$ (assumindo que os vértices são acessíveis de forma eficiente, por exemplo, como o primeiro elemento de uma lista de adjacência ou através de um índice).
    \item \textbf{Busca em Profundidade (DFS) e Pós-Ordem:} Uma DFS padrão em um grafo com $n$ vértices e $m$ arestas tem complexidade $O(n+m)$. A computação da sequência de pós-ordem é uma parte natural da DFS (um vértice é adicionado à lista de pós-ordem quando a chamada recursiva da DFS para esse vértice é concluída) e não adiciona sobrecarga significativa à complexidade da DFS.
    \item \textbf{Definição da ordenação de remoção:} Atribuir a sequência de pós-ordem $P_1, \dots, P_n$ à ordenação $x_1, \dots, x_n$ envolve a manipulação de uma lista de $n$ vértices, o que leva $O(n)$ tempo.
\end{enumerate}
Somando as complexidades das etapas, a complexidade total do algoritmo é dominada pela etapa da DFS, resultando em $O(n+m)$.

\end{document}