\documentclass{article}
\usepackage[utf8]{inputenc}

\title{MAC0459 - Q5}
\author{Leonardo  Heidi Almeida Murakami}
\date{December 2021}

\begin{document}

\maketitle

\section{Perguntas da Auto-avaliação}
\begin{enumerate}
\item Você assistiu a todas as aulas online? Se não, o que a/o fez não assistir?
%\vspace{2.0cm}
\item Como você avalia seu entendimento das aulas ao conteúdo? O fato das aulas serem online afetaram seu entendimento positiva, ou negativamente?
%\vspace{2.0cm}
\item Você participou das tarefas em grupo realizadas em sala de aulas?  Se não, o que a/o fez não participar? Como você avalia a sua participação nas discussões do grupo? Como você avalia a ferramenta (discord) utilizada para as interações?
%\vspace{2.0cm}
\item Você procurou conhecer mais profundamente algum método exposto em aulas? Lembra quais? 
Qual a sua avaliação, em termos de entendimento, desse seu estudo?
%\vspace{2.0cm}
\item Você teve motivação para vir às aulas e participar das discussões? Se sim, o que aumentaria 
ainda mais sua motivação? Se não, o que você sugere para que as aulas sejam motivadoras?
%\vspace{2.0cm}
\item A disciplina satisfez suas expectativas? Faça uma análise crítica dos objetivos
  pretendidos por você ao se matricular e dos objetivos que você alcançou. 
%\vspace{2.0cm}
\item Além do que foi dado, o que mais você gostaria de aprender? Existe algum tópico você gostaria 
que fosse aprofundado?
%\newpage
\item A quantidade e a dificuldade das tarefas práticas foi adequada e suficiente? Você
  sente-se seguro para continuar estudando o assunto sozinho? O que poderia ser melhorado nas 
tarefas? Avalie para cada uma das tarefas.
\item Como você avalia a inserção de palestras de convidados na disciplina? Escreva um parágrafo sobre o que foi abordado em cada uma das duas palestras e sua avaliação sobre a adequabilidade no contexto da disciplina.
%\vspace{3.0cm}
\item Faça uma breve auto-avaliação de seu desempenho considerando sua 
         facilidade com o conteúdo da disciplina, seu entendimento dos 
         conceitos, sua assiduidade às aulas, sua
         participação em aulas e seu desempenho nas tarefas. Se você se
         sentir confortável para isso, atribua-se uma nota de 0 a 10
         de acordo com sua auto-avaliação. Essa nota será considerada
         no cômputo final da sua nota, caso o professor entenda que
         você soube se auto-avaliar.

\end{enumerate}

\section{Respostas da Auto-avaliação}
\begin{enumerate}
\item Não assisti a todas as aulas online, devido a algumas reuniões do trabalho acabei não conseguindo participar de todas as aulas, porém, participei da grande maioria.
%\vspace{2.0cm}
\item Já possuía conhecimento pelo menos básico de todo o conteúdo apresentado fora o Neo4J, se tivesse que avaliar o meu entendimento das aulas em uma nota de 0 a 10 seria capaz de me dar pelo menos um 8. A razão da nota não ser o valor cheio seria a dificuldade inicial de entender a base de dados do Neo4J (além de achar que sempre é possível melhorar, logo não me daria uma 10).

Dito isso, acho que o curso ser online afetou meu entendimento (na soma de todos os fatores) mais positivamente que negativamente devido a capacidade de fazer pesquisas e código juntamente as aulas, tornando uma experiência mais "hands-on" que acabava sendo mais prazerosa que as outras aulas do semestre, por exemplo.
%\vspace{2.0cm}
\item Não consegui participar da primeira atividade em grupo pois estava em processo preparatório para uma cirurgia. 

Reconheço que sou uma pessoa um pouco mais quieta e gostaria de ter tido uma participação mais ativa na discussão do grupo, mas ainda sim acredito ter sido de uso para o grupo.

O Discord foi uma ferramenta facilitadora muito boa para interações, seja no horário de aula ou fora do horário de aula
%\vspace{2.0cm}
\item Lembro de ter procurado bem mais profundamente sobre o Neo4J por ter achado um conceito extremamente interessante e pouco valorizado, lembro também de ter estudado sobre os modelos de árvore mais profundamente após a aula por ser utilizado na empresa em que eu trabalho.
Avalio este meu estudo, baseado em entendimento, como muito bom. Embora não tenha entendido muito tecnicamente o uso do Neo4J de forma completa por exemplo, fui capaz de absorver muito mais da sua motivação de uso e como ser utilizado de forma correta
%\vspace{2.0cm}
\item Sim! Era uma das aulas que possuía mais expectativas durante a semana para ver, acho que aulas com mais atividades em grupo ou com conteúdo hands-on acabariam me motivando mais ainda a participar das aulas por acabar gerando mais dúvidas e ser mais aplicada as coisas do mundo real.
%\vspace{2.0cm}
\item Sim, possuía um certo receio de me matrícular em uma matéria de Ciência (e Engenharia) de Dados por ja ter tido experiências ruins com o tópico na faculdade. Dito isso, meus objetivos eram: conseguir me aprofundar mais em engenharia de dados, aumentar o conhecimento teórico de modelagem, aumentar o conhecimento de métricas de predição e sinto que todos foram alcançados.
Após a semana de SQL, que, embora muito boa, já possuía o conhecimento passado durante a aula, fiquei receoso de que não atingiria tão bem o primeiro objetivo, embora após a aula de Neo4J este tenha sido realizado muito bem.
%\vspace{2.0cm}
\item Gostaria de desenvolver melhor o conhecimento teórico em Redes Neurais e entender os modelos estado da arte de deep learning (por exemplo, Bert, HrNETV2, entre outros), gostaria de ter tido um aprofundamento maior em SQL, passando sobre conteúdos como Window functions, otimização de queries ou SQL Dinâmico.
%\newpage
\item Assim como citado em cima, acho que poderíamos ter mais tarefas práticas/aulas práticas durante todo o semestre, dito isso não acho que acabou detraindo para minha capacidade de estudar o assunto sozinho.
\item Sobre a palestra do Caio Lente, ja conhecia bastante sobre a linguagem R por ter estudado ela por algum tempo e gostei bastante de algumas informações novas levantadas por ele, como por exemplo, o uso da linguagem C++ de dentro do próprio R, mas, ainda assim vejo problemas na linguagem (assim como vejo em todas as outras) que me incentivam mais a utilizar Python, por exemplo. Gostei bastante dos pontos levantados e pelo meu conhecimento prévio acabei ficando bem interessado nos pontos de comparação que ele trouxe, achei uma palestra super adequada ao contexto da disciplina, dado que R tenta ser uma linguagem 100\% integrada com dados.

Sobre a palestra da mestra Mariane Gonzales, achei bem interessante a palestra dada por ela para poder ver a stack que o banco Pan utiliza, fiquei inicialmente receoso por ter um preconceito negativo com palestras empresariais em aulas, utilizadas em sua maior parte para capturar novos talentos, mas fui surpreendido muito positivamente. Embora a parte que mais aproveitei a palestra foi para comparar a stack do banco Pan com a stack de onde trabalho, acho que pode ter agregado bastante sobre como a ciencia (e engenharia) de dados funciona no mundo real, fora de um notebook.

Normalmente não gosto muito de palestras inseridas durante disciplinas, porém fui surpreendido pelas palestras coesas desta disciplina.
%\vspace{3.0cm}
\item Dito tudo que foi falado aqui, gostaria de me atribuir uma nota 7 em minha auto-avaliação. Embora não tenha conseguido participar de uma atividade em grupo (que, como dito acima, senti falta de mais oportunidades na disciplina) sinto que consegui extrair uma boa parte do conhecimento que foi passado pelo professor, sempre que necessário ou que possuía, retirava eventuais dúvidas por meio do slido e, por fim, sinto que foi uma das poucas disciplinas na faculdade até hoje, incrivelmente junto a de Algebra Linear, que acabei por aprender com o professor invés de apenas por conta própria

\end{enumerate}

Gostaria de aproveitar este espaço para agradecer pelo semestre professor Hirata e monitor Artur! Foi uma das disciplinas que mais gostei na faculdade até hoje e espero que continue sendo ministrada deste jeito (ou melhor!) pelos próximos anos em que ciência de dados se torne cada vez mais relevante para a sociedade e cada vez mais buscada nas universidades.

\end{document}
