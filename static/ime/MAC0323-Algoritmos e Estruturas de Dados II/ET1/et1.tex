\documentclass[11pt,reqno,a4paper]{amsart}

\usepackage{setspace}
\usepackage[portuguese]{babel}
\usepackage[utf8]{inputenc} % Or use \usepackage{inputenc}[utf8] if needed by editor

\usepackage{fullpage}
\usepackage{amsmath}
\usepackage{amssymb} % For symbols like \lfloor, \rfloor, \min, \max
\usepackage{amsthm} % Recommended for theorem environments

% Consistent spacing for displayed math
\usepackage{mathtools} 
\mathtoolsset{showonlyrefs=false} % Optional: only number referenced equations

\def\Assin#1{\noindent\textit{Assinatura}\strut\\%:\\
\framebox[\textwidth]{\phantom{\vrule height#1}}}

\newtheoremstyle{prova_style}{}{}{\itshape}{}{\bfseries}{.}{ }{}
\theoremstyle{prova_style}
\newtheorem*{prova}{Prova} % Unnumbered proof environment

% Define lg for logarithm base 2
\DeclareMathOperator{\lgop}{lg}
\newcommand{\lgfn}[1]{\lgop #1}

\title{\textsl{
    MAC0323 Algoritmos e Estruturas de Dados II}\\\vspace{3\jot}
  Folha de solução}
\author[MAC0323 Folha de solução]{} % Author is optional or can be filled

\date{13 de Abril de 2025} % Use \date command

\begin{document}
\parindent=0pt

\maketitle
\thispagestyle{empty} 
\pagestyle{plain}
\onehalfspace

\textbf{Nome:} Leonardo Heidi Almeida Murakami\enspace \hfill \enspace
\textbf{NUSP:} 11260186\enspace\hfill

\medskip
\Assin{1cm}

\medskip \textit{Sua assinatura atesta a autenticidade e
  originalidade de seu trabalho e que você se compromete a seguir o
  código de ética da USP em suas atividades acadêmicas, incluindo esta
  atividade.}

\bigskip
\textbf{Exercício:} Exercício Teórico I - MAC0323\enspace\hfill\enspace
%\textbf{Data:} 13/04/2025\enspace % Date is already in \date command

\bigskip
\noindent\textbf{SOLUÇÃO}

\section{Exercício 1}
Seja $T$ uma Árvore Binária Completa (ABC) com $N$ nós, onde $N \geq 1$. Seja $N = (b_n b_{n-1} \dots b_0)_2$ a representação binária de $N$, onde $b_n=1$ (e portanto, $n = \lfloor \lgfn N \rfloor$). Sejam $T_1$ e $T_2$ as subárvores esquerda e direita da raiz de $T$, com $N_1$ e $N_2$ nós, respectivamente.

Queremos provar que se $b_{n-1}=1$, então $N_1 = 2^n - 1$ e $N_2 = (b_{n-1}\dots b_0)_2$.

\begin{prova}
\textbf{Definições e Propriedades da ABC:}
\begin{itemize}
    \item Uma ABC com $N$ nós tem altura $h = n = \lfloor \lgfn N \rfloor$.
    \item Todos os níveis $0, 1, \dots, n-1$ estão completamente preenchidos, contendo $2^n - 1$ nós.
    \item O nível $n$ (último nível) contém os $k = N - (2^n - 1) = N - 2^n + 1$ nós restantes, preenchidos da esquerda para a direita.
    \item O número total de nós é $N = 1 (\text{raiz}) + N_1 + N_2$.
    \item A representação binária de $N$ é $N = \sum_{i=0}^{n} b_i 2^i$. Como $b_n=1$, temos $N = 2^n + \sum_{i=0}^{n-1} b_i 2^i = 2^n + (b_{n-1} \dots b_0)_2$.
    \item O número de nós no último nível é $k = N - 2^n + 1 = (2^n + (b_{n-1} \dots b_0)_2) - 2^n + 1 = (b_{n-1} \dots b_0)_2 + 1$.
\end{itemize}

\textbf{Tamanhos das Subárvores $N_1$ e $N_2$:}
A subárvore esquerda $T_1$ consiste nos nós da subárvore esquerda completa até o nível $n-1$, mais os nós que "caem" na subárvore esquerda no último nível.
\begin{itemize}
    \item Nós em $T_1$ dos níveis $1$ a $n-1$ (relativos a $T$): $\frac{(2^n - 1) - 1}{2} = 2^{n-1} - 1$.
    \item Nós em $T_1$ do nível $n$ (último nível de $T$): Os primeiros $\min(k, 2^{n-1})$ nós do último nível $k$ vão para a subárvore esquerda (pois a capacidade da subárvore esquerda no nível $n$ é $2^{n-1}$).
    \item Portanto, $N_1 = (2^{n-1} - 1) + \min(k, 2^{n-1})$.
\end{itemize}
A subárvore direita $T_2$ consiste nos nós da subárvore direita completa até o nível $n-1$, mais os nós restantes do último nível.
\begin{itemize}
    \item Nós em $T_2$ dos níveis $1$ a $n-1$: $2^{n-1} - 1$.
    \item Nós em $T_2$ do nível $n$: Os nós restantes $\max(0, k - 2^{n-1})$.
    \item Portanto, $N_2 = (2^{n-1} - 1) + \max(0, k - 2^{n-1})$.
\end{itemize}

\textbf{Caso $b_{n-1}=1$:}
Se $b_{n-1}=1$, então o valor $(b_{n-1} \dots b_0)_2$ é no mínimo $2^{n-1}$.
\begin{itemize}
    \item $(b_{n-1} \dots b_0)_2 = 1 \cdot 2^{n-1} + (b_{n-2} \dots b_0)_2 \geq 2^{n-1}$.
    \item Consequentemente, $k = (b_{n-1} \dots b_0)_2 + 1 \geq 2^{n-1} + 1$. Logo, $k > 2^{n-1}$.
\end{itemize}1
Agora, calculamos $N_1$ e $N_2$ sob esta condição:
\begin{itemize}
    \item $N_1 = (2^{n-1} - 1) + \min(k, 2^{n-1})$. Como $k > 2^{n-1}$, $\min(k, 2^{n-1}) = 2^{n-1}$.\\
          $N_1 = (2^{n-1} - 1) + 2^{n-1} = 2 \cdot 2^{n-1} - 1 = \boxed{2^n - 1}$.
          (Isto significa que a subárvore esquerda é uma árvore binária completa de altura $n-1$).
    \item $N_2 = (2^{n-1} - 1) + \max(0, k - 2^{n-1})$. Como $k > 2^{n-1}$, $\max(0, k - 2^{n-1}) = k - 2^{n-1}$.\\
          $N_2 = (2^{n-1} - 1) + (k - 2^{n-1}) = k - 1$.
    \item Substituindo $k = (b_{n-1} \dots b_0)_2 + 1$: \\
          $N_2 = ((b_{n-1} \dots b_0)_2 + 1) - 1 = \boxed{(b_{n-1} \dots b_0)_2}$.
\end{itemize}
Assim, provamos que se $b_{n-1}=1$, então $N_1 = 2^n - 1$ e $N_2 = (b_{n-1} \dots b_0)_2$.
\end{prova}

\newpage

\section{Exercício 2}
Usando a mesma notação do Exercício 1, queremos provar que se $b_{n-1}=0$, então $N_1=(1b_{n-2}\dots b_0)_2$ e $N_2=2^{n-1}-1$.

\begin{prova}
Utilizamos as mesmas definições e as fórmulas gerais para $N_1$ e $N_2$ derivadas no Exercício 1:
\begin{itemize}
    \item $N = (1 b_{n-1} \dots b_0)_2$, $n = \lfloor \lgfn N \rfloor$.
    \item $k = N - 2^n + 1 = (b_{n-1} \dots b_0)_2 + 1$.
    \item $N_1 = (2^{n-1} - 1) + \min(k, 2^{n-1})$.
    \item $N_2 = (2^{n-1} - 1) + \max(0, k - 2^{n-1})$.
\end{itemize}

\textbf{Caso $b_{n-1}=0$:}
Se $b_{n-1}=0$, então o valor $(b_{n-1} \dots b_0)_2 = (0 b_{n-2} \dots b_0)_2 = (b_{n-2} \dots b_0)_2$.
\begin{itemize}
    \item $(b_{n-1} \dots b_0)_2 = \sum_{i=0}^{n-2} b_i 2^i < 2^{n-1}$.
    \item Consequentemente, $k = (b_{n-1} \dots b_0)_2 + 1 \leq (2^{n-1} - 1) + 1 = 2^{n-1}$.
\end{itemize}
Agora, calculamos $N_1$ e $N_2$ sob esta condição:
\begin{itemize}
    \item $N_1 = (2^{n-1} - 1) + \min(k, 2^{n-1})$. Como $k \leq 2^{n-1}$, $\min(k, 2^{n-1}) = k$. \\
          $N_1 = (2^{n-1} - 1) + k$.
    \item Substituindo $k = (b_{n-1} \dots b_0)_2 + 1$: \\
          $N_1 = (2^{n-1} - 1) + ((b_{n-1} \dots b_0)_2 + 1) = 2^{n-1} + (b_{n-1} \dots b_0)_2$.
    \item Como $b_{n-1}=0$, $(b_{n-1} \dots b_0)_2 = (b_{n-2} \dots b_0)_2$. \\
          $N_1 = 2^{n-1} + (b_{n-2} \dots b_0)_2$.
    \item Este valor é precisamente a representação binária $\boxed{(1 b_{n-2} \dots b_0)_2}$. (O $1$ está na posição $n-1$).
    \item $N_2 = (2^{n-1} - 1) + \max(0, k - 2^{n-1})$. Como $k \leq 2^{n-1}$, $\max(0, k - 2^{n-1}) = 0$. \\
          $N_2 = (2^{n-1} - 1) + 0 = \boxed{2^{n-1} - 1}$.
          (Isto significa que a subárvore direita é uma árvore binária completa de altura $n-2$).
\end{itemize}
Assim, provamos que se $b_{n-1}=0$, então $N_1 = (1 b_{n-2} \dots b_0)_2$ e $N_2 = 2^{n-1} - 1$.
\end{prova}

\newpage

\section{Exercício 3}
Seja $T$ uma ABC com $N$ nós. Seja $h(x)$ a altura do nó $x$ em $T$ (definida como o comprimento do caminho mais longo de $x$ até uma folha na subárvore enraizada em $x$). Seja $S_N = \sum_{x \in T} h(x)$ a soma das alturas de todos os nós em $T$. Seja $u_N$ o número de bits 1 na representação binária de $N$. Queremos provar que $S_N = N - u_N$.

\begin{prova}
Procederemos por indução forte sobre o número de nós $N$.

\textbf{Base da Indução:} Para $N=1$.
\begin{itemize}
    \item A árvore $T$ consiste apenas no nó raiz $r$.
    \item A representação binária é $N=(1)_2$.
    \item O número de bits 1 é $u_1 = 1$.
    \item A altura da raiz (que também é folha) é $h(r)=0$.
    \item A soma das alturas é $S_1 = h(r) = 0$.
    \item A fórmula $N - u_N$ resulta em $1 - u_1 = 1 - 1 = 0$.
    \item Como $S_1 = N - u_N$, a base da indução é válida.
\end{itemize}

\textbf{Hipótese Indutiva (HI):} Assuma que para todo $k$ tal que $1 \le k < N$, a soma das alturas $S_k$ em uma ABC com $k$ nós satisfaz $S_k = k - u_k$.

\textbf{Passo Indutivo:} Considere uma ABC $T$ com $N > 1$ nós. Seja $r$ a raiz de $T$. Sejam $T_1$ e $T_2$ as subárvores esquerda e direita com $N_1$ e $N_2$ nós, respectivamente, onde $N = 1 + N_1 + N_2$.
\begin{itemize}
    \item A altura da raiz $r$ é $h(r) = n = \lfloor \lgfn N \rfloor$.
    \item A soma das alturas em $T$ pode ser calculada recursivamente: A altura de cada nó $x$ na subárvore $T_i$ (onde $i=1$ ou $i=2$) contribui para a soma $S_N$. A soma das alturas dos nós dentro de $T_1$ (calculadas como se $T_1$ fosse uma árvore independente) é $S_{N_1}$. Similarmente para $T_2$, a soma é $S_{N_2}$. A altura da raiz $h(r)$ deve ser adicionada.
    \item Assim, a relação recursiva é: $S_N = h(r) + S_{N_1} + S_{N_2}$.
    \item Note que $N_1 \ge 0$ e $N_2 \ge 0$. Como $N > 1$, pelo menos uma subárvore não é vazia. Se $N_1=0$ (ou $N_2=0$), então $S_{N_1}=0$ (ou $S_{N_2}=0$), e a HI ainda se aplica (formalmente, $u_0=0$, então $S_0=0-0=0$). Como $N \ge 1$, temos $N_1 < N$ e $N_2 < N$.
    \item Aplicando a Hipótese Indutiva para $S_{N_1}$ e $S_{N_2}$ (assumindo $N_1, N_2 \ge 1$, ou tratando $S_0=0, u_0=0$ se $N_1$ ou $N_2$ for 0):
      \[ S_N = n + (N_1 - u_{N_1}) + (N_2 - u_{N_2}) \]
    \item Usando $N_1 + N_2 = N - 1$:
      \[ S_N = n + (N - 1) - (u_{N_1} + u_{N_2}) \]
\end{itemize}
Queremos mostrar que $S_N = N - u_N$. Para isso, precisamos verificar se a seguinte igualdade é verdadeira:
\[ n + (N - 1) - (u_{N_1} + u_{N_2}) = N - u_N \]
Rearranjando os termos, a igualdade acima é equivalente a:
\[ u_N = u_{N_1} + u_{N_2} - n + 1 \]
Vamos verificar esta relação usando os resultados dos Exercícios 1 e 2. Seja $N = (1 b_{n-1} \dots b_0)_2$. O número de bits 1 em $N$ é $u_N = 1 + \sum_{i=0}^{n-1} b_i$.

\textbf{Caso 1: $b_{n-1} = 1$.}
\begin{itemize}
    \item Do Exercício 1, $N_1 = 2^n - 1 = (11\dots1)_2$ ($n$ uns) e $N_2 = (b_{n-1} \dots b_0)_2 = (1 b_{n-2} \dots b_0)_2$.
    \item $u_{N_1} = n$.
    \item $u_{N_2}$ é o número de bits 1 em $(b_{n-1} \dots b_0)_2$, que é $\sum_{i=0}^{n-1} b_i$.
    \item Verificando a relação:
      $u_{N_1} + u_{N_2} - n + 1 = n + \left(\sum_{i=0}^{n-1} b_i\right) - n + 1 = 1 + \sum_{i=0}^{n-1} b_i$.
    \item Este resultado é exatamente $u_N = 1 + \sum_{i=0}^{n-1} b_i$. A relação é válida neste caso.
\end{itemize}

\textbf{Caso 2: $b_{n-1} = 0$.}
\begin{itemize}
    \item Do Exercício 2, $N_1 = (1 b_{n-2} \dots b_0)_2$ e $N_2 = 2^{n-1} - 1 = (11\dots1)_2$ ($n-1$ uns).
    \item $u_{N_1}$ é o número de bits 1 em $(1 b_{n-2} \dots b_0)_2$. O bit mais significativo (posição $n-1$) é 1. Os outros bits são $b_{n-2}, \dots, b_0$. Então $u_{N_1} = 1 + \sum_{i=0}^{n-2} b_i$.
    \item $u_{N_2} = n-1$.
    \item Verificando a relação:
      $u_{N_1} + u_{N_2} - n + 1 = \left(1 + \sum_{i=0}^{n-2} b_i\right) + (n-1) - n + 1$
      $= 1 + \sum_{i=0}^{n-2} b_i + n - 1 - n + 1 = 1 + \sum_{i=0}^{n-2} b_i$.
    \item Agora, vejamos $u_N$. Como $N = (1 b_{n-1} \dots b_0)_2$ e $b_{n-1}=0$:
      $u_N = 1 + b_{n-1} + \sum_{i=0}^{n-2} b_i = 1 + 0 + \sum_{i=0}^{n-2} b_i = 1 + \sum_{i=0}^{n-2} b_i$.
    \item Novamente, o resultado $u_{N_1} + u_{N_2} - n + 1$ é igual a $u_N$. A relação é válida neste caso também.
\end{itemize}

Como a relação $u_N = u_{N_1} + u_{N_2} - n + 1$ foi verificada em ambos os casos possíveis para $b_{n-1}$, a equação $S_N = n + (N - 1) - (u_{N_1} + u_{N_2})$ se simplifica corretamente para $S_N = N - u_N$.

\textbf{Conclusão da Indução:} Pelo princípio da indução forte, provamos que para qualquer ABC $T$ com $N \ge 1$ nós, a soma das alturas de todos os nós é $S_N = N - u_N$.
\end{prova}

\end{document}