\documentclass{article}
\usepackage[utf8]{inputenc}
\usepackage[portuguese]{babel}
\usepackage{amsmath}
\usepackage{graphicx}
\usepackage{listings}
\usepackage{hyperref}
\usepackage[left=2cm,right=2cm,top=2cm,bottom=2cm]{geometry}

\makeatletter
\renewcommand\paragraph{\@startsection{paragraph}{4}{\z@}%
  {-3.25ex\@plus -1ex \@minus -.2ex}%
  {1.5ex \@plus .2ex}%
  {\normalfont\normalsize\bfseries}}
\makeatother

\title{Análise do Sistema de Gerenciamento de Contatos\\
\large Estruturas de Dados e Complexidade}
\author{Relatório Técnico}
\date{\today}

\begin{document}

\maketitle

\section{Introdução}
Este relatório apresenta uma análise detalhada do sistema de gerenciamento de contatos (Contatinho), focando nas estruturas de dados utilizadas, complexidade computacional, análise de memória e comparações com sistemas de banco de dados relacionais.

\section{Estruturas de Dados}
\subsection{Visão Geral}
O sistema utiliza quatro estruturas principais:
\begin{itemize}
    \item Red-black BST para indexação por nome
    \item Red-black BST para indexação por Instagram
    \item Red-black BST com listas encadeadas para indexação por prioridade
    \item Fila para manter ordem de inserção
\end{itemize}

\subsection{Justificativa das Escolhas}
\subsubsection{Red-black BST para Nome e Instagram}
\paragraph{Vantagens:}
\begin{itemize}
    \item Busca, inserção e remoção em $O(\log n)$
    \item Manutenção automática da ordem dos elementos
    \item Garantia de unicidade das chaves
\end{itemize}

\paragraph{Alternativas Consideradas:}
\begin{itemize}
    \item Hash Tables: Descartada por não manter ordenação
    \item Lista: Descartada por busca $O(n)$
    \item Lista Ligada: Descartada por busca $O(n)$        
    \item Árvore AVL: Descartada por ter maior overhead de implementação e manutenção, com benefícios marginais de performance para o caso de uso
\end{itemize}

\subsubsection{Red-black BST com Listas para Prioridade}
\paragraph{Vantagens:}
\begin{itemize}
    \item Ordenação automática das prioridades
    \item Suporte a múltiplos contatos por prioridade através de listas encadeadas
    \item Acesso eficiente $O(\log n)$
\end{itemize}

\paragraph{Alternativas Consideradas:}
\begin{itemize}
    \item Heap: Complexo para gerenciar múltiplos elementos com mesma prioridade
    \item Lista Ordenada: Inserção ineficiente $O(n)$
    \item Árvore AVL: Complexidade desnecessária para dados não únicos
\end{itemize}

\subsubsection{Fila para Ordem de Chegada}
\paragraph{Vantagens:}
\begin{itemize}
    \item Implementação natural do princípio FIFO
    \item Remoção do primeiro elemento em $O(1)$
    \item Overhead mínimo de memória
    \item Simplicidade de implementação
\end{itemize}

\paragraph{Alternativas Consideradas:}
\begin{itemize}
    \item Lista Encadeada: Funcionalidade excessiva para o requisito
    \item Pilha: Ordem LIFO não atenderia o requisito
    \item Lista Circular: Complexidade desnecessária
\end{itemize}

\section{Análise de Complexidade}
\subsection{Complexidade Temporal}
\paragraph{Adicionar Contato: $O(\log n)$}
A operação de adição envolve:
\begin{itemize}
    \item Inserção na árvore de nomes: $O(\log n)$
    \item Inserção na árvore de Instagram: $O(\log n)$
    \item Inserção na árvore de prioridades: $O(\log n)$
    \item Inserção na fila: $O(1)$
\end{itemize}
A complexidade final é $O(\log n)$. Todas as operações são independentes e executadas sequencialmente.

\paragraph{Buscar Contato: $O(\log n)$}
A busca é realizada em duas etapas:
\begin{itemize}
    \item Busca na árvore de nomes: $O(\log n)$
    \item Se não encontrado, busca na árvore de Instagram: $O(\log n)$
\end{itemize}
Mesmo no pior caso (quando o contato não existe), a complexidade mantém-se em $O(\log n)$ pois as buscas são sequenciais.

\paragraph{Atualizar Contato: $O(\log n)$}
A atualização consiste em:
\begin{itemize}
    \item Busca do contato na árvore de nomes: $O(\log n)$
    \item Remoção da antiga referência na árvore de Instagram: $O(\log n)$
    \item Inserção da nova referência na árvore de Instagram: $O(\log n)$
    \item Atualização na árvore de prioridades: $O(\log n)$ para remoção e inserção
\end{itemize}
Como todas as operações são executadas sequencialmente, mantendo a complexidade final em $O(\log n)$.

\subsubsection{Operações de Listagem}

\paragraph{Listar por Prioridade: $O(n \log n)$}

\begin{itemize}
    \item Percorrer a árvore de prioridades: $O(\log n)$ por nível
    \item Para cada prioridade, acessar a lista de contatos: $O(k)$ onde $k$ é o número de contatos naquela prioridade
    \item Como $\sum k = n$, e precisamos percorrer $O(\log n)$ níveis da árvore
    \item A complexidade total resulta em $O(n \log n)$
\end{itemize}

\paragraph{Listar em Ordem Alfabética: $O(n)$}
Operação simples:
\begin{itemize}
    \item Percorrer a árvore em ordem (in-order traversal): $O(n)$
    \item Cada nó é visitado exatamente uma vez
    \item Não requer ordenação adicional pois a árvore já se mantêm ordenada
\end{itemize}

\paragraph{Remover Primeiro Contato: $O(1)$}
A remoção envolve:
\begin{itemize}
    \item Acesso ao primeiro elemento da fila: $O(1)$
    \item Remoção nas árvores: $O(\log n)$ para cada árvore
    \item Porém, como temos acesso direto ao elemento a ser removido, a complexidade amortizada é $O(1)$
\end{itemize}

\subsubsection{Análise do Pior Caso}
Em cenários específicos, algumas operações podem ter desempenho diferente:
\begin{itemize}
    \item Inserções ordenadas poderiam degradar uma árvore não balanceada para $O(n)$
    \item Múltiplos contatos com mesma prioridade podem aumentar o tempo de listagem
    \item Remoções consecutivas podem causar rebalanceamento em cascata
\end{itemize}

\subsubsection{Otimizações Possíveis}
Algumas melhorias poderiam ser implementadas:
\begin{itemize}
    \item Cache de resultados frequentes
    \item Lazy deletion para remoções
    \item Bulk operations para operações em lote
    \item Estruturas auxiliares para casos específicos de uso
\end{itemize}

\subsection{Análise de Memória}
\subsubsection{Estruturas Principais}
Consumo de memória por contato:
\begin{itemize}
    \item Objeto Contato: $O(1)$ - campos fixos
    \item Referências em TreeMaps: $O(1)$ por índice
    \item Lista na prioridade: $O(k)$ onde k é número de contatos com mesma prioridade
\end{itemize}

\subsubsection{Overhead Total}
\begin{itemize}
    \item Espaço total: $O(n)$ para n contatos
    \item Overhead de indexação: $O(n)$ devido às estruturas auxiliares
\end{itemize}

\end{document}