\documentclass[11pt,reqno,a4paper]{amsart}

\usepackage{setspace}
\usepackage[portuguese]{babel}
\usepackage[utf8]{inputenc}

\usepackage{fullpage}
\usepackage{setspace}

\def\Assin#1{\noindent\textit{Assinatura}\strut\\%:\\
\framebox[\textwidth]{\phantom{\vrule height#1}}}

\begin{document}
\parindent=0pt

\title{\textsl{
    MAC0121 Algoritmos e Estruturas de Dados I}\\\vspace{3\jot}
  Folha de solução}
\author[MAC0122 Folha de solução]{}

\maketitle
\thispagestyle{empty} 
\pagestyle{plain}
\onehalfspace

\textbf{Nome:} Leonardo Heidi Almeida Murakami\enspace \hfill \enspace
\textbf{NUSP:}11260186\enspace\hfill

\medskip
\Assin{1cm}

\medskip \textit{Sua assinatura atesta a autenticidade e
  originalidade de seu trabalho e que você se compromete a seguir o
  código de ética da USP em suas atividades acadêmicas, incluindo esta
  atividade.}

\bigskip
\textbf{Exercício:} Exercicio Teórico I - MAC0121\enspace\hfill\enspace
\textbf{Data:} 03/10/2024\enspace

\bigskip
\noindent\textbf{SOLUÇÃO}

\section{Exercício 1}

\subsection{Explicação do funcionamento do algoritmo}

\begin{answer} 
\ 
\\
\\
Podemos dividir o funcionamento do algoritmo da seguinte forma
\\
\\
\begin{math}
    \text{funcao}(n) = 
\begin{cases} 
n - 10 & \text{para } n > 100 \\
\text{funcao}(\text{funcao}(n + 11)) & \text{para } n \leq 100
\end{cases}
\end{math}
\\
\\
Para qualquer valor de $n$ maior que 100, temos que a funcao retornara $n-10$, para outros valores, a funcao crescerá o valor de $n$ até atingir o valor de 101 e retornar como valor final 91.

\end{answer}

\subsection{Provando que o caso base é atingido para qualquer inteiro n}
\
\\
\\
Para provar que esta função sempre chegará a seu caso base, provaremos que, para qualquer valor $n$ inicial, depois de um numero finito de chamadas recursivas, que este valor atingirá um valor superior a 100
\begin{enumerate}
    \item Primeiro devemos notar que, para qualquer $ n > 100 $, temos o caso base, dado que a função retorna um valor imediatamente
    \item Para $n \leq 100$, vamos verificar o que acontece em cada chamada recursiva:
    \begin{itemize}
        \item Na chamada interna, $n$ é acrescido de 11.
        \item Se este valor for $\leq 100$, será acrescido novamente de 11 na próxima chamada.
        \item Este valor é decrescido, numa chamada, de no máximo 10
        \item Logo, n sempre terá seu valor, previsivelmente crescente e será $ > 100$ após uma quantidade finita de iterações
    \end{itemize}
    \item Logo, sabemos que, existe um $k$ tal que $n + 11k > 100$ para $n < 100$
\end{enumerate}
\\

\subsection{Calculando o numero de chamadas para $funcao(n)$}
\
\\
\\
Podemos concluir, a partir do funcionamento da função, que a quantidade de chamadas para a $funcao$ pode ser calculada pela seguinte formula
\\
\\
\begin{math}
    \text{chamadas}(n) = 
\begin{cases} 
1 & \text{para } n > 100 \\
\text{chamadas}(n + 1) + 2 & \text{para } n \leq 100
\end{cases}
\end{math}


\newpage
\section{Exercício 2}
\subsection{Calculando f(1,6)}
\
\\
\\
Vamos analisar o comportamento da função para \( \text{funcao}(1, 6) \):

\begin{verbatim}
public static int funcao(int a, int b) {
   if (b == 0)
      return 0;
   else
      return (a + funcao(a, b-1));
}    
\end{verbatim}

A função é recursiva e tem o seguinte comportamento:
\begin{itemize}
  \item Quando \( b = 0 \), a função retorna 0.
  \item Caso contrário, ela soma \( a \) ao valor retornado pela chamada recursiva \( \text{funcao}(a, b-1) \).
\end{itemize}

Agora, vamos simular a execução para \( \text{funcao}(1, 6) \):
\begin{itemize}
    \item \( \text{funcao}(1, 6) \) chama \( \text{funcao}(1, 5) \) e adiciona 1 ao resultado.
    \item \( \text{funcao}(1, 5) \) chama \( \text{funcao}(1, 4) \) e adiciona 1 ao resultado.
    \item \( \text{funcao}(1, 4) \) chama \( \text{funcao}(1, 3) \) e adiciona 1 ao resultado.
    \item \( \text{funcao}(1, 3) \) chama \( \text{funcao}(1, 2) \) e adiciona 1 ao resultado.
    \item \( \text{funcao}(1, 2) \) chama \( \text{funcao}(1, 1) \) e adiciona 1 ao resultado.
    \item \( \text{funcao}(1, 1) \) chama \( \text{funcao}(1, 0) \), que retorna 0 (caso base).
\end{itemize}


Agora, a função começa a retornar os valores somados: 
\begin{itemize}
   \item \( \text{funcao}(1, 1) \) retorna \( 1 + 0 = 1 \).
   \item \( \text{funcao}(1, 2) \) retorna \( 1 + 1 = 2 \).
   \item \( \text{funcao}(1, 3) \) retorna \( 1 + 2 = 3 \).
   \item \( \text{funcao}(1, 4) \) retorna \( 1 + 3 = 4 \).
   \item \( \text{funcao}(1, 5) \) retorna \( 1 + 4 = 5 \).
   \item \( \text{funcao}(1, 6) \) retorna \( 1 + 5 = 6 \).
\end{itemize}
Portanto, o resultado de \( \text{funcao}(1, 6) \) é $6$.
\\
\\
\subsection{Provando que a função termina para qualquer a e b}

Vamos analisar o comportamento da função:
\begin{itemize}
    \item O caso base ocorre quando \( b = 0 \), onde a função retorna 0 e não realiza mais chamadas recursivas.
    \item Em cada chamada recursiva, o valor de \( b \) é decrementado em 1 (\( b - 1 \)).
\end{itemize}
Assim, independentemente do valor de \( a \), o valor de \( b \) sempre diminui até atingir o caso base \( b = 0 \). Como \( b \) diminui a cada iteração e é um valor inteiro, eventualmente \( b \) chegará a 0, o que faz com que a função termine para qualquer valor de b. Como atingir o caso base independe do valor de $a$, a função sempre termina.\
\newpage
\section{Exercício 3}
\subsection{Código para calcular a sequencia de Fibonacci}
\
\\
\\
Utilizando a formula dada, temos que:
\\
\begin{verbatim}
public class Main {
    public static int fibonacci(int n) {
        if (n == 1 || n == 2) {
            return 1;
        }
        if (n % 2 != 0) {
            int k = (n + 1) / 2;
            int fk1 = fibonacci(k);
            int fk2 = fibonacci(k - 1);
            return fk1 * fk1 + fk2 * fk2;
        } else {
            int k = n / 2;
            int fk1 = fibonacci(k + 1);
            int fk2 = fibonacci(k - 1);
            return fk1 * fk1 - fk2 * fk2;
        }
    }

    public static void main(String[] args) {
        if (args.length == 0) {
            System.out.println("Por favor, passe o valor de N como argumento.");
            return;
        }

        int N = Integer.parseInt(args[0]);
        for (int i = 1; i <= N; i++) {
            System.out.println("F(" + i + ") = " + fibonacci(i));
        }
    }
}
\end{verbatim}

Este código nos permite calcular até o numero 46 da sequencia de Fibonacci (de valor 1836311903), isto ocorre devido ao overflow. Se alterarmos o código para usar o BigInteger do java de forma que o código pareça algo como:
\newpage
\begin{verbatim}
import java.math.BigInteger;

public class Main {
    public static BigInteger fibonacci(int n) {
        // Caso base
        if (n == 1 || n == 2) {
            return BigInteger.ONE;
        }
        if (n % 2 != 0) {
            int k = (n + 1) / 2;
            BigInteger fk1 = fibonacci(k);
            BigInteger fk2 = fibonacci(k - 1);
            return fk1.multiply(fk1).add(fk2.multiply(fk2));
        } else {
            int k = n / 2;
            BigInteger fk1 = fibonacci(k + 1);
            BigInteger fk2 = fibonacci(k - 1);
            return fk1.multiply(fk1).subtract(fk2.multiply(fk2));
        }
    }

    public static void main(String[] args) {
        if (args.length == 0) {
            System.out.println("Por favor, insira o valor de N como argumento.");
            return;
        }

        int N = Integer.parseInt(args[0]);
        for (int i = 1; i <= N; i++) {
            System.out.println("F(" + i + ") = " + fibonacci(i));
        }
    }
}
\end{verbatim}
Com este código foi possível calcular o valor ate o numero 50000 da sequencia de Fibonacci, de valor aproximado de $1.84e+4083$ (calculado usando outro programa)
\\
\\
\newpage
\subsection{Prova que o de Djikstra é equivalentes a versão original}
\subsubsection{Prova da Equivalência da Definições Alternativa de Fibonacci}
\
\\
\\
Vamos provar por indução que as seguintes definições alternativas da função de Fibonacci são equivalentes à definição original:

\textbf{Base da indução:}
Para $n = 0$ e $n = 1$, as definições são idênticas à original, então a equivalência é trivialmente verdadeira.
\\
\textbf{Hipótese indutiva:}
Assumimos que a equivalência é verdadeira para todos os valores até $n$, onde $n \geq 2$. Além disso, assumimos que as seguintes equações são verdadeiras:
\\
\begin{align*}
F_{2n-1} &= F_n^2 + F_{n-1}^2 \\
F_{2n} &= F_n(F_{n+1} + F_{n-1})
\end{align*}

\textbf{Passo indutivo:}
Precisamos provar que a equivalência se mantém para $n+1$. Temos dois casos a considerar:

\textit{Caso 1:} $(2n+1)$ (caso onde n é impar)

Provamos $F_{2n+1} = F_n^2 + F_{n+1}^2$:

\begin{align*}
F_{2n+1} &= F_{2n}+F_{2n-1} \\
&= F_{n}(F_{n+1}+F_{n-1}) + F_{n-1}^2+ F_n^2 \\
&= F_{n-1}(F_{n-1}+F_n)+F_n^2+F_nF_{n+1} \\
&= F_{n+1}^2+F_{n}^2.
\end{align*}

\textit{Caso 2:} $(2n+2)$ (caso onde n é par)

Provamos $F_{2n+2} = F_{n+1}(F_{n+2} + F_n)$:

\begin{align*}
F_{2n+2} &= F_{2n+1} + F_{2n} \\
&= (F_{n+1}^2 + F_n^2) + F_n(F_{n+1} + F_{n-1}) \\
&= F_{n+1}^2 + F_n^2 + F_n F_{n+1} + F_n F_{n-1} \\
&= F_{n+1}(F_{n+1} + F_n) + F_n(F_n + F_{n-1}) \\
&= F_{n+1}F_{n+2} + F_n F_{n+1} \\
&= F_{n+1}(F_{n+2} + F_n)
\end{align*}

\textbf{Conclusão:}
Provamos que a definição de djikstra é equivalente à definição original para $n+1$, assumindo que são equivalentes para $n$. Pelo princípio da indução matemática, as definições são equivalentes para todos os números naturais.

\endgroup
\end{document}


