\documentclass{article}
\usepackage[a4paper,top=3cm,bottom=2cm,inner=3cm,outer=2cm]{geometry}
\usepackage[singlespacing]{setspace}
\usepackage{graphicx} 

\title{AUP2409 - Resumo}
\author{Leonardo  Heidi Almeida Murakami}
\date{November 2024}

\begin{document}

\maketitle
\newpage

\tableofcontents
\newpage

\section{Coisas que nos fazem Designers}

\begin{itemize}
    \item Domínio completo do método de projeto especifico do Design
    \item Elevada sofisticação formal, refinamento, elegância, apuro formal, bom gosto e sensibilidade para a boa forma
    \item Conhecimentos teóricos e conceituais do Design
    \item Empatia profunda com a perspectiva do usuário
\end{itemize}

\newpage
\section{Limites entre o Design e a Arquitetura}

\textbf{Similaridades:}
\begin{itemize}
    \item Ambos pertencem a categoria de artes aplicadas
    \item Ambos lidam com a interface entre a tecnologia e humanos (em aspectos diferentes)
    \item Ambos envolvem a dimensão de projeto
\end{itemize}

\textbf{Diferenças:}
\begin{enumerate}
    \item Foco e proposito
    \begin{itemize}
        \item Arquitetura: Organização e sistematização dos espaços
        \item Design: Suporte das atividades, acoes e tarefas
    \end{itemize}
    \item Aproximação da forma
    \begin{itemize}
        \item Arquitetura: Foca na contraforma (continentes)
        \item Design: Foca na forma (conteúdo)
    \end{itemize}
    \item Escala e Dimensão
    \begin{itemize}
        \item Arquitetura: Escala muito maior
        \item Design: Escala menor mais próxima ao corpo humano
    \end{itemize}
    \item Língua do Design
    \begin{itemize}
        \item Arquitetura: Predominantemente reta, ortogonal, com vértices e curvas fechadas
        \item Design: Geralmente orgânica, suave e antropomórfica
    \end{itemize}
    \item Processo de Produção
    \begin{itemize}
        \item Arquitetura: Processo de construção
        \item Design: Processo de produção através da manufatura e gráfica
    \end{itemize}
    \item Relação com o usuário
    \begin{itemize}
        \item Arquitetura: Normalmente usuários definidos, conhecidos e específicos
        \item Design: Usuários desconhecidos, universais e genéricos
    \end{itemize}
    \item Influencia Regional
    \begin{itemize}
        \item Arquitetura: Mais suscetível a condições regionais e restrições
        \item Design: Mais transregional e internacional
    \end{itemize}
    \item Escala de Produção
    \begin{itemize}
        \item Arquitetura: Larga escala mas com produção menos frequente
        \item Design: Maior frequência de produção
    \end{itemize}
    \item Mobilidade
    \begin{itemize}
        \item Arquitetura: Itens fixos
        \item Design: Itens moveis
    \end{itemize}
    \item Dependência material
    \begin{itemize}
        \item Arquitetura: Maior dependência nos materiais locais e processos
        \item Design: Mais independente da seleção material
    \end{itemize}
    \item Responsabilidade Técnica
    \begin{itemize}
        \item Arquitetura: Responsável pelos cálculos e pela operacionalização do projeto
        \item Design: Normalmente delega as responsabilidades técnicas para a engenharia
    \end{itemize}
    \item Pedagogia (no brasil)
    \begin{itemize}
        \item Arquitetura: Associado a escola de Paris, mais formalista
        \item Design: Associado a escola de Ulm, mais funcionalista
    \end{itemize}
    \item Metodologia de Projeto
    \begin{itemize}
        \item Arquitetura: Design baseado na experiencia
        \item Design: Design baseado no experimento
    \end{itemize}
\end{enumerate}

\newpage
\section{Limites entre o Design e as Artes Visuais}
\textbf{Diferenças:}
\begin{enumerate}
    \item Função e Propósito
    \begin{itemize}
        \item Artes Visuais: Independe de funcionalidade e utilidade pratica
        \item Design: Depende da funcionalidade e necessidade do usuário
    \end{itemize}
    \item Processo criativo
    \begin{itemize}
        \item Artes Visuais: Método caixa preta, mais liberdade, menos restrições
        \item Design: Método caixa de cristal, múltiplas restrições e limitações
    \end{itemize}
    \item Abordagem
    \begin{itemize}
        \item Artes Visuais: Subjetivo, focado no mundo interior, auto-expressão
        \item Design: Objetivo, focado no mundo exterior, solução de problemas
    \end{itemize}
    \item Estilo de trabalho
    \begin{itemize}
        \item Artes Visuais: Genialidade individual, estilo pessoal
        \item Design: Trabalho em equipe, normalmente anonimo
    \end{itemize}
    \item Linguagem Visual
    \begin{itemize}
        \item Artes Visuais: Complexo, arbitrário, ambíguo, forma não modular
        \item Design: Sintético, modular, estruturado, limpo, formas imediatas
    \end{itemize}
    \item Comunicação
    \begin{itemize}
        \item Artes Visuais: Múltiplas camadas de significado, aberto para interpretação
        \item Design: Direto, objetivo e com um significado claro
    \end{itemize}
\end{enumerate}

\newpage
\section{Limites entre o Design e a Ilustração}

\textbf{Similaridades:}
\begin{itemize}
    \item Ambos envolvem a reprodução digital/mecânica
    \item Ambos possuem conteúdo informacional
    \item Ambos servem propósitos específicos
    \item Ambos alcançam múltiplos usuários através da reprodução em serie
\end{itemize}

\textbf{Diferenças:}
\begin{enumerate}
    \item Estilo
    \begin{itemize}
        \item Ilustração: Estilo pessoal proeminente
        \item Design: Apessoal, abordagem neutra
    \end{itemize}
    \item Expressão
    \begin{itemize}
        \item Ilustração: Mais figurativo e expressivo
        \item Design: Mais abstrato e sintético
    \end{itemize}
    \item Complexidade
    \begin{itemize}
        \item Ilustração: Menos restrições e fatores do projeto
        \item Design: Restrições simultâneas e considerações
    \end{itemize}
    \item Integração
    \begin{itemize}
        \item Ilustração: Podem estar contido num projeto de Design
        \item Design: Não pode estar contido na ilustração
    \end{itemize}
    \item Abordagem Visual
    \begin{itemize}
        \item Ilustração: Mais analógica, diferenciado, arbitrário
        \item Design: Mais digital, contrastado, programado, econômico
    \end{itemize}
\end{enumerate}

\newpage
\section{Limites entre o Design e o Artesanato Artístico-utilitário}
\textbf{Similaridades:}
\begin{itemize}
    \item Ambos pertencem a categoria de artes aplicadas
    \item Ambos servem propósitos funcionais e suprem necessidades humanas
\end{itemize}
\textbf{Diferenças:}
\begin{enumerate}
    \item Criação e produção
    \begin{itemize}
        \item Artesanato: Unidade entre criador e produtor
        \item Design: Separação do Designer e produtor
    \end{itemize}
    \item Padrão de produção
    \begin{itemize}
        \item Artesanato: Pode variar dependendo da situação
        \item Design: Segue modelos invariáveis e fixos
    \end{itemize}
    \item Natureza do Processo
    \begin{itemize}
        \item Artesanato: Padrões analógicos, progressão continua
        \item Design: Padrões digitais, posições predeterminadas
    \end{itemize}
    \item Controle da produção
    \begin{itemize}
        \item Artesanato: Possível interferir durante a manufatura
        \item Design: Não é possível interferir quando já estiver em produção
    \end{itemize}
    \item Metodologia
    \begin{itemize}
        \item Artesanato: Mais pratico, manual, empírico e intuitivo 
        \item Design: Mais planejado, disciplinado, metódico e racional
    \end{itemize}
    \item Foco do conhecimento
    \begin{itemize}
        \item Artesanato: Conhecimento especifico de materiais e técnicas
        \item Design: Conhecimento generalista de vários materiais e processos da industria
    \end{itemize}
    \item Investimento e escala
    \begin{itemize}
        \item Artesanato: Escala menor, menor investimento, produção local
        \item Design: Maior escala, investimento significante e produção centralizada
    \end{itemize}
    \item Relação com o usuário
    \begin{itemize}
        \item Artesanato: Normalmente funciona com usuários conhecidos
        \item Design: Usuários universais e desconhecidos
    \end{itemize}
    \item Educação
    \begin{itemize}
        \item Artesanato: Aprendizado através da relação mestre-aprendiz, autodidata
        \item Design: Educação superior formal
    \end{itemize}
    \item Complexidade de Produção
    \begin{itemize}
        \item Artesanato: Técnicas mais simples de manufaturas, artefatos menos complexos
        \item Design: Processos industriais mais sofisticados, artefatos mais complexos
    \end{itemize}
    \item Pesquisa e Planejamento
    \begin{itemize}
        \item Artesanato: Menos sistemático, abordagem mais circunstancial
        \item Design: Complexo, rigoroso e pesquisa e planejamento sistemático
    \end{itemize}
    \item Variedade Material
    \begin{itemize}
        \item Artesanato: Leque limitado de materiais e técnicas
        \item Design: Amplo leque de materiais e processos de produção
    \end{itemize}
\end{enumerate}
\newpage
\section{Limites entre o Design e a Engenharia}
\begin{enumerate}
    \item Função principal
    \begin{itemize}
        \item Engenharia: Fornece o "como" (operacional, viabilidade técnica)
        \item Design: Fornece o "o que" (interface com o usuário, contexto geral)
    \end{itemize}
    \item Estrutura relacional
    \begin{itemize}
        \item Engenharia: Prove a base tecnológica contida no amplo projeto de Design
        \item Design: Cria a interface entre a tecnologia e os humanos
    \end{itemize}
    \item Flow da tecnologia
    \begin{itemize}
        \item Engenharia: Pode ser puxado (Design demanda da engenharia)
        \item Design: Pode ser empurrado (engenharia oferece ao Design)
    \end{itemize}
\end{enumerate}
\newpage
\section{Limites entre o Design e o TI (Tecnologia da Informação)}
\begin{enumerate}
    \item Divisão de cargos
    \begin{itemize}
        \item TI: Habilita a operação técnica
        \item Design: Cria a interface do usuário e a experiencia
    \end{itemize}
    \item Problemas metodológicos
    \begin{itemize}
        \item Confusão entre métodos ágeis (baixa inercia) e processos de Design (alta inercia)
        \item Desapropriação de termos como "front-end Design" e "back-end Design"
    \end{itemize}
    \item Outros problemas
    \begin{itemize}
        \item Confusão crescente entre a geração de conteúdo e o Design
        \item Mal uso da terminologia do Design no contexto do TI
        \item Ma-aplicação do conceito de "wicked problems" para o Design
    \end{itemize}
\end{enumerate}
\newpage
\section{Limites entre o Design e a Publicidade}
\begin{enumerate}
    \item Linguagem
    \begin{itemize}
        \item Publicidade: Linguagem persuasiva, sedutora com foco nos consumidores
        \item Design: Linguagem neutra, informativa e impessoal para servir usuários
    \end{itemize}
    \item Considerações Éticas
    \begin{itemize}
        \item Separação histórica dos campos esta ficando mais borrada
        \item Design algumas vezes atuando a serviço da publicidade (ex: Design Gráfico)
        \item "Styling" como a forma da publicidade penetrar no Design de produto
    \end{itemize}
    \item Problemas profissionais
    \begin{itemize}
        \item Agencias de publicidade com influencia na esfera politica acabam afetado o trabalho do Design
        \item Diluição do repertorio técnico e cultural do Design
        \item Perda de qualidade técnica decorrente da reserva do mercado do Design gráfico
    \end{itemize}
\end{enumerate}
\newpage
\section{Limites entre o Design e o Marketing}
\begin{enumerate}
    \item Foco
    \begin{itemize}
        \item Marketing: Enfase nas vendas e comercialização
        \item Design: Enfase em servir os usuários
    \end{itemize}
    \item Perspectiva do usuários
    \begin{itemize}
        \item Marketing: Vê as pessoas como consumidores (meios)
        \item Design: Vê as pessoas como usuários (fins)
    \end{itemize}
    \item Relação
    \begin{itemize}
        \item O Marketing pode guiar a atividade do Design no desenvolvimento de produtos
        \item O Marketing pode prover dados valiosos para a pesquisa do Design
        \item O Marketing historicamente liderava o Design de serviços ate por volta de 1990
    \end{itemize}
\end{enumerate}
\newpage
\section{O conceito de funcionalidade do Design}
A funcionalidade é considerada como a uma das dimensões mais essenciais do Design. Sem utilidade pratica, não existiria o Design.
\begin{itemize}
    \item A funcionalidade é aquilo que distingue o Design das artes plásticas
    \item Se manifesta tanto no Design visual, Design de produto e no Design de serviços
    \item Esta atrelado a pragmática, semiótica e ergonomia
    \item De acordo com Bernd Löbach, o Design tem três funções: a prática, a estética e simbólica
    \item A funcionalidade não é uma criação humana, é um padrão natural
    \item A natureza, em si, é funcionalista, crescendo de dentro para fora (a evolução, por si só, é funcionalista em princípio)
\end{itemize}
\subsection{Perspectivas Criticas}
\begin{itemize}
    \item Se o Design perdesse sua dimensão funcional, seria necessário criar um novo termo para a atividade de fazer objetos funcionais
    \item Existe uma tensão entre o funcionalismo e o formalismo
    \item A Função não deve ser tiranicamente excludente com outros aspectos, mas, também não deve ser abandonada sem cuidados
\end{itemize}
\subsection{O Design}
\begin{itemize}
    \item O Design é, fundamentalmente, sobre resolucionar problemas e atingir necessidades
    \item Desenvolver boas soluções técnicas e funcionais requer rigor e uma competência técnica extraordinária
    \item Design deveria ser centrada no usuário invés de centrada no Designer
    \item A funcionalidade não exclui a estética - assim como boa comida deveria ser tanto nutritiva quanto gostosa
    \item O Design requer um entendimento do mundo externo, suas condições, usuários e suas necessidades
\end{itemize}
\subsection{Conclusão (sobre a funcionalidade)}
Enquanto o Design sem estética pode ser feio e o Design sem consideração semântica pode ser inapropriado, o Design sem funcionalidade pode nem ser Design

\newpage
\section{O conceito de estética no Design}
\subsection{Fundação Filosófica da Estética no Design}
\begin{itemize}
    \item Enraizada no conceito grego de qualidade, excelência e virtuosidade
    \item Tensão gerada entre estética centrada no objeto e centrada no observador
    \item Questões do belo universal ou do belo construído socialmente.
    \begin{itemize}
        \item Polarização indiscutível de preferencias estéticas ao redor de alguns padrões sugere certa universalidade
        \item Respostas a estética não estão distribuídas regularmente entre as possibilidades, elas se concentram em certos padrões
    \end{itemize}
\end{itemize}
\subsection{A Relação Funcional-Estética}
\begin{itemize}
    \item A beleza no Design não é uma característica adicionada, mas sim está intrinsecamente ligada a função
    \item Três possibilidades de relações
    \begin{enumerate}
        \item O Belo como a função em si
        \item O Belo como algo separado da função
        \item O Belo como o resultado natural de uma boa função
    \end{enumerate}
    \item O belo é apenas mais uma funcionalidade no meio de várias outras, invés de:
    \begin{itemize}
        \item Uma consideração a priori que domina todos os outros aspectos
        \item Um pensamento puramente decorativo feito depois
    \end{itemize}
\end{itemize}
\subsection{O Fenômeno Kitsch}
Definido através de uma serie sistemática de observações sócio-econômicas:
\begin{itemize}
    \item Começa com a desigualdade econômica
    \item Leva ao diferente acesso a cultura material
    \item Ocorre que a cultura material das classes mais altas tende a ser mais sofisticada, confortável, bela, inspiradora e atraente
    \item Cria uma aspiração das classes mais baixas em direção a estética das classes mais altas
    \item A tecnologia permite uma imitação parcial dos objetos de elite
    \item Resulta in "Kitsch" como uma forma de imitação cultural
\end{itemize}
Não necessariamente é sobre "mal gosto", mas mais sobre um processo socio-econômico onde:
\begin{itemize}
    \item As classes mais baixas desejam acesso aos marcadores culturais da elite
    \item Conseguem acessar estes apenas parcialmente ou a imitações
    \item Estas imitações focam apenas nas características superficiais aceitas por todos
\end{itemize}
\newpage
\section{O conceito de módulo no Design}
\subsection{Definição}
Um módulo é uma unidade auto-contida, repetível na estrutura de um sistema, que deve ser desacoplável, transferível, substituível, padronizado e regular. Ele cria uma continuidade interna (e descontinuidade externa) e funciona como "átomos" do Design - blocos de Lego que podem ser combinados
\subsection{A Relação Analógica-Digital}
\begin{itemize}
    \item O Modulo converte variações analógicas em passos/níveis
    \item Exemplos:
    \begin{itemize}
        \item Cinto com um ajuste continuo vs. Um com buracos pré feitos
        \item Taxi (Análogo/continuo) vs Ônibus (digital/discreto)
    \end{itemize}
    \item Deve ser criado um equilíbrio entre a flexibilidade e a padronização
\end{itemize}
\subsection{Implicações}
\begin{itemize}
    \item Permite a produção em massa e eficiência industrial
    \item Permite:
    \begin{itemize}
        \item Multiconfigurabilidade
        \item Intercambiabilidade
        \item Reconfigurabilidade
        \item Simplificação da manutenção e reparo
        \item Padronização das partes
    \end{itemize}
    \item Reduz a variabilidade para aumentar a escalabilidade
\end{itemize}
\subsection{Estética e Princípios Estruturais}
\begin{itemize}
    \item Cria unidade na diversidade
    \item Estabelece um ritmo e uma coerência
    \item Funciona como um DNA da forma
    \item Prove um sistema de referencia interno
    \item Equilíbrio entre a frequência (numero de passos) e amplitude (tamanho dos passos)
\end{itemize}
\subsection{Implicações Sociais e Econômicas}
\begin{itemize}
    \item Permite a produção em massa dos sistemas
    \item Democratiza o acesso a produtos devido a baixa dos preços (pela eficiência de produção)
    \item Cria tensão entre a customização e a padronização
    \item Impacta em como as pessoas interagem com o Design dos objetos
\end{itemize}
\subsection{Impacto no Processo de Design}
\begin{itemize}
    \item Requer um planejamento sistemático invés de um desenvolvimento intuitivo
    \item Mudança de paradigma de algo mais artesanal/analógico para um pensamento digital/industrial
    \item Cria sistemas controláveis e previsíveis
\end{itemize}
\subsection{Limitações e Desafios}
\begin{itemize}
    \item Pode criar rigidez no processo de criar o Design de soluções
    \item Pode não se encaixar perfeitamente em todas as situações
    \item Necessita do equilíbrio entre a padronização e flexibilidade
    \item Cria cenários para fases "favoráveis" e "desfavoráveis" quando o digital encontra o analógico
\end{itemize}

\section{O conceito de proporções e boa forma no Design}
\subsection{A Natureza das proporções no Design}
\begin{itemize}
    \item Fundamental relação entre a proporção e:
    \begin{itemize}
        \item Sintaxe (relação de união e diversidade)
        \item Estética (beleza, elegância, harmonia)
        \item Funcionalidade
        \item Percepção Humana
    \end{itemize}
    \item Não é sobre colocar informações espalhadas arbitrariamente
    \item Requer competência, rigor e senso estético
\end{itemize}
\subsection{Proporções Significantes e Uteis no Design}
\begin{itemize}
    \item Proporção Áurea ($\phi \approx 1.618$)
    \item Quadrado ($1:1$)
    \item Duplo Quadrado ($1:2$)
    \item Raiz de 2 ($\sqrt{2}$) - Retângulo DIN e ABNT
    \item Raiz de 3 ($\sqrt{3}$)
    \item Raiz de 5 ($\sqrt{5} = (2*\phi) + 1$
    \item Proporção 2/3 (e 1/3) - Utilizada em telas, fotografias
    \item Proporção 3/4 (e 1/4)
    \item Proporção 10/16 - Muito próximo a proporção áurea
    \item Proporção 9/16 - Monitores de computador
\end{itemize}

\subsection{Características da Boa Forma}
\begin{itemize}
    \item Não pode ser esticada nem comprimida
    \item Livre de elementos arbitrários
    \item Planejada e estável
    \item Modular e proporcional
    \item Com uma estrutura limpa e coerente
    \item Abordagem sistemática
    \item Livre de barulho visual 
\end{itemize}

\subsection{Aplicações Práticas no Design}
\begin{itemize}
    \item Layouts de página e tipografia
    \item Design de interface e tela
    \item Proporções de produto
    \item Elementos de arquitetura
    \item Hierarquia visual
    \item Margens e espaços
\end{itemize}

\subsection{Metodologia do Design}
\begin{itemize}
    \item Abordagem dedutiva invés de indutiva
    \item Planejando do geral para o especifico
    \item Consideração sistemática das relações proporcionais
    \item Integração com os sistemas modulares
    \item Importância de estudar e testar as proporções
    \item Menos sobre formulas mecânicas e mais sobre aplicações conscientes e inteligentes
\end{itemize}
\newpage

\section{O conceito de semiótica do Design}
\subsection{Estrutura central da semiótica do Design}
\begin{itemize}
    \item Divida em três áreas principais:
    \begin{itemize}
        \item Semântica (significado)
        \item Sintática (relação entre os elementos)
        \item Pragmática (interpretação pratica e uso)
    \end{itemize}
    \item Deve ser entendida como algo pratico e útil invés de algo super teórico
    \item Foca em como as coisas comunicam e não apenas naquilo que elas comunicam
\end{itemize}
\subsection{O conceito de semântica no Design}
\begin{itemize}
    \item Funciona através de significantes (elementos observáveis) e significados
    \item Funciona como uma ponte entre a forma física e os conceitos abstratos
    \item Exemplos:
    \begin{itemize}
        \item A identidade visual de um banco passando a ideia de confiança
        \item O Design de um celular sugerindo juventude
        \item O Design de um carro expressando masculinidade ou feminilidade 
    \end{itemize}
    \item Podem ser diretas (pictográficas) ou indiretas (ideográficas)
    \item Devem equilibrar a essência e a aparência
    \item Possui vantagens evolucionarias para o entendimento rápido
    \item Pode ser ma-utilizada para propósitos enganosos ("Kitsch")
    \item Surge nas frases do dia a dia:
    \begin{itemize}
        \item "Este bule \underline{me lembra} um cisne"
        \item "Esta fonte \underline{parece} letra de criança"
    \end{itemize}
\end{itemize}
\subsection{O conceito de sintática no Design}
\begin{itemize}
    \item Lida com a relação entre elementos do Design
    \item Foca na coerência e unidade interna
    \item Envolve:
    \begin{itemize}
        \item Proporções
        \item Modularização
        \item Harmonia
        \item Ritmo visual
        \item Relações estruturadas
    \end{itemize}
    \item Frequentemente negligenciada na educação contemporânea do Design
    \item Surge nas perguntas do dia a dia: 
    \begin{itemize}
        \item "Destoa muito?"
        \item "Fica bem?"
        \item "Combina?"
    \end{itemize}
\end{itemize}
\subsection{O conceito da pragmática do Design}
\begin{itemize}
    \item Se preocupa em como os usuários interpretam e interagem com o design no contexto
    \item Relacionada com a funcionalidade básica e usabilidade
    \item Ainda esta sub-desenvolvida na teoria do design
    \item Conectada a ergonômica e ao entendimento do usuário
    \item Foca nas funções primarias e mais essenciais
    \item Surge nas perguntas do dia a dia: 
    \begin{itemize}
        \item \underline{O que se faz com} um sorvete?
        \item Essa placa \underline{serve para o que?}
    \end{itemize}
\end{itemize}
\subsection{Considerações gerais da Semiótica e a Pratica do Design}
\begin{itemize}
    \item A pratica do design deve considerar todos os 3 aspectos simultaneamente
    \item Deve ter um equilíbrio entre a função e a aparência
    \item Deve evitar significados superficiais ou enganosos
    \item Precisa de integração com os métodos de produção
\end{itemize}
\subsection{Considerações Éticas}
\begin{itemize}
    \item Tensão entre o significado autentico e o enganoso
    \item Responsabilidade cultural e social na mensagem
    \item Equilíbrio entre a necessidade do marketing e a comunicação honesta
\end{itemize}
\subsection{Contexto Histórico}
\begin{itemize}
    \item A escola de Ulm foi a pioneira no tema de semióticas do design
    \item A mudança do modernismo ao pós-modernismo afeta a abordagem 
    \item Evolução da compreensão semântica no entendimento de diferentes campos do design
\end{itemize}

\section{O conceito de projeto e método em design}
\subsection{Processos Dedutivos vs. Indutivos}
\begin{itemize}
    \item O Design usa, principalmente, processos dedutivos (de cima para baixo)
    \item Contrastes como jardins Franceses (planejados) vs. jardins Ingleses (orgânicos)
    \item Movimento de todo para partes vs. de partes para o todo
    \item Planejamento a longo prazo vs. resolução imediata de problemas
\end{itemize}
\subsection{Conceitos fundamentais da metodologia do Design}
\begin{itemize}
    \item Projeto (do latim "projectum"): algo lançado para frente
    \item Método (do Grego "methodos"): um caminho de chegar do ponto A para o ponto B
    \item Distinções Chave:
    \begin{itemize}
        \item Entre o método (caminho) e as ferramentas (técnicas)
        \item Entre a substantiva ("o que") da técnica ("como")
        \item Entre método e metodologia (estudo dos métodos)
    \end{itemize}
\end{itemize}
\subsection{Elementos chave do Processo do Design}
\begin{itemize}
    \item Uma abordagem sistemática e centralizadora
    \item Baseada em fatores externos do mundo
    \item Funciona igualmente no Design de Produtos, Design Visual e Design de Serviços
    \item Inclui múltiplas fases:
    \begin{itemize}
        \item Definição do problema
        \item Coleta de informações e analise
        \item Definição de pre-requisitos do projeto
        \item Geração de alternativas
        \item Desenvolvimento e comunicação
    \end{itemize}
    \item Características importantes:
    \begin{itemize}
        \item Trabalho em equipe
        \item Desenho e representação baseada no modelo
        \item Necessidade de testes do usuário e verificação
        \item Justificativas racionais para todas as decisões
    \end{itemize}
\end{itemize}
\subsection{Considerações Metodológicas}
\begin{itemize}
    \item Aspectos Essenciais:
    \begin{itemize}
        \item O problema sempre vem primeiro
        \item É necessário entender o contexto
        \item É necessário modelos físicos e virtuais
        \item É necessário um equilíbrio em estrutura e flexibilidade
        \item Utilizar da verificação continua com os usuários
    \end{itemize}
    \item Pontos Criticos:
    \begin{itemize}
        \item Distinção entre a caixa de cristal (design) e a caixa preta (artes plásticas)
        \item Evolução de aberto para fechado, quantidade para qualidade
        \item A importância de não pular etapas
        \item A importância de verificação com os usuários para validar os cenários imaginados
    \end{itemize}
\end{itemize}


\end{document}
